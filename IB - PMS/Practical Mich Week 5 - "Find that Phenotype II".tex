\documentclass[
]{article}
\usepackage{amsmath,amssymb}


\title{IB PMS Practical Week 5: `\,`Find that Phenotype II"}
\author{BGN: 54}
\date{}

\begin{document}
\maketitle

\section{Photosynthetic Gas Exchange (Licor) - RbcS
mutants}\label{photosynthetic-gas-exchange-licor---rbcs-mutants}

\subsection{Hypotheses}\label{hypotheses}

\begin{quote}
\emph{How do you expect the assimilation (A) and evaporation (E) rates
to vary between wt and ssu2? Why?}
\end{quote}

It is expected that there will be a decrease in both the assimilation
and evaporation levels of the plant in the ssu2 mutant. This is due to
the decrease in the levels of {} due to decreased levels of RuBisCo
activity.

\begin{quote}
\emph{How do you expect water use efficiency (WUE) to vary between wt
and ssu2? Why?}
\end{quote}

We expect the water use efficiency in the mutant to decrease due to the
reduced carbon assimilated per unit of water.

\begin{quote}
\emph{What variation do you expect to measure in stomatal conductance {}
and internal {}, {} between wt and ssu2? Why?}
\end{quote}

I would expect no difference between the mutant and wild type as this
would be an indirect effect in which multiple factors would affect the
gas flow. For example, the stomata might be open due to the signalling
of assimilation deficiencies and thus an increased level of stomatal
conductance and an increased internal carbon dioxide concentration.
However, this increase in internal carbon dioxide levels might lead to
the closing of the stomata as the guard cells would sense the higher
levels of internal carbon dioxide due to the lack of assimilation.

\begin{quote}
\emph{From the A/ci responses, how do you expect {} and {} to vary
between wt and ssu2? Why?}
\end{quote}

{}: we expect a decrease in the rubisco quantity in the mutant, and so
the efficiency of fixing will decrease. Thus {} will decrease.\\
{}: we expect less respiration to take place in the mutant as a function
of less carbon fixation, decreasing the levels of glucose available for
respiration as well as a signalling response to decrease the levels of
respiration in the aim of preserving the plant\textquotesingle s carbon
stores.

\subsection{Description of results}\label{description-of-results}

\begin{itemize}
\tightlist
\item
  From the A/Ci fitted model, we can see that there is a decrease in the
  carbon dioxide assimilation levels at each internal carbon dioxide
  concentration level in the ssu2 mutant when compared to the wild type
  ({}). This supports the hypothesis made.

  \begin{itemize}
  \tightlist
  \item
    The diagnostic plots look okay; just a slight concern could be
    raised around the residuals vs leverage where points 6 and 12 have
    quite high leverage on the model.
  \end{itemize}
\item
  From the intercellular CO2 concentrations (Ci), we can see that the
  ssu2 had an increase in Ci with an average of 375.17 umol CO2/mol
  compared to the wild type, where the Ci was at 285.33 umol CO2/mol
  ({}). This does not support our hypothesis.

  \begin{itemize}
  \tightlist
  \item
    The diagnostic plots show that the data is not homogeneously spread
    around the mean fitted values between the ssu2 and wt. However,
    there is no cause for significant concern about this.
  \end{itemize}
\item
  From the stomatal conductance to water vapour, there appears to be no
  significant correlation between the water conductance and the ssu2
  gene ({}).

  \begin{itemize}
  \tightlist
  \item
    The diagnostic plots raise no concern.
  \end{itemize}
\item
  From the CO2 assimilation data, we can see that there is a large
  increase in the assimilation of carbon dioxide in wt compared to ssu2
  with an average of 12.7 umol CO2/m2/s and 3.09 umol CO2/m2/s,
  respectively ({}).

  \begin{itemize}
  \tightlist
  \item
    The diagnostic plots show that the data is not homogeneously spread
    around the mean fitted values between the ssu2 and wt. However,
    there is no cause for significant concern about this.
  \end{itemize}
\item
  From the transpiration data there is no difference in the
  transpiration levels between the ssu2 mutant and the wildtype ({}).

  \begin{itemize}
  \tightlist
  \item
    The diagnostic plots raise no concern.
  \end{itemize}
\item
  From the intrinsic water use efficiency data the wildtype has a large
  increase in water use efficiency with over three times the efficiency
  compared to the ssu2 mutant ({}).

  \begin{itemize}
  \tightlist
  \item
    There are slightly higher residuals for the wildtype dataset;
    however, this is no direct cause of concern as the more extreme
    residuals are along the 1:1 on the normalQ-Q plot.
  \end{itemize}
\end{itemize}

\section{Light use - RbcS Mutants (PAM fluorometer and Chlorophyll
meter)}\label{light-use---rbcs-mutants-pam-fluorometer-and-chlorophyll-meter}

\subsection{Hypothesis}\label{hypothesis}

\begin{quote}
\emph{How do you expect chlorophyll content to vary between wt and ssu2?
Why?}
\end{quote}

We expect a decrease in the chlorophyll content in the mutant as a
feedback mechanism to prevent the waste of the plant\textquotesingle s
resources.

\begin{quote}
\emph{How do you expect maximal {} (Fv/Fm) to vary between wt and ssu2?
Why?}
\end{quote}

We do not expect the maximal {} to change, as this is independent of
RuBisCo and its activity.

\begin{quote}
\emph{Do you expect to find a difference in ETR between wt and ssu2?
Why?}
\end{quote}

We expect the ETR to decrease in ssu2 due to a slower return of the
proton receptors (NAD and FAD), which would lead to a backing up of the
electron transport chains in photosynthesis.

\begin{quote}
Do you expect to find a difference in NPQ between wt and ssu2? Why?
\end{quote}

We expect a decrease in the levels of NPQ in ssu2 and, thus an increase
in fluorescence to counter the reduction in NPQ ({}, and if PC is
constant and {}). This occurs when there is not enough light to trigger
photochemistry.

\subsection{Description of results}\label{description-of-results-1}

\begin{itemize}
\tightlist
\item
  The relative chlorophyll content is higher in the wildtype compared to
  the ssu2 mutant by just over 50\% ({}).

  \begin{itemize}
  \tightlist
  \item
    The diagnostic plots show that the data is not homogeneously spread
    around the mean fitted values between the ssu2 and wt. However,
    there is no cause for significant concern about this.
  \end{itemize}
\item
  There is a increase of about 2x of the electron transport rate in the
  wild type when compared to the ssu2 mutant ({}).

  \begin{itemize}
  \tightlist
  \item
    The diagnostic plots show no cause for concern over the data.
  \end{itemize}
\item
  There appears to be no difference in the maximal PSII quantum
  efficiency between the wildtype and ssu2 mutant ({}).

  \begin{itemize}
  \tightlist
  \item
    The diagnostic plots show that the data does not follow the 1:1 line
    on the normalQ-Q plot, suggesting that there might be a better model
    if a transformation was performed on the data.
  \end{itemize}
\item
  There appears to be an outlier with one of the wildtype data points;
  however, the data suggests that there is no relationship between the
  wildtype and the ssu2 mutant and the NPQ amounts. It might be worth
  collecting more data on this and confirming the outlier.
\item
  The PSII operating efficiency is over double in the wildtype when
  compared with the ssu2 mutant ({}).

  \begin{itemize}
  \tightlist
  \item
    The diagnostic plots show no cause for concern.
  \end{itemize}
\end{itemize}

\section{Light harvesting - PsbS Mutant (chlorophyll fluorescence
imager)}\label{light-harvesting---psbs-mutant-chlorophyll-fluorescence-imager}

\subsection{Hypothesis}\label{hypothesis-1}

\begin{quote}
\emph{Which genotype would you expect to have the highest NPQ? Why?}
\end{quote}

We would expect the highest levels of NPQ in the over-expressor of PsbS.
This is due to PsbS being a key protein in the regulation and action of
NPQ.

\begin{quote}
\emph{Do you expect to find differences in maximal {} (Fv/Fm) between
these three genotypes? Why?}
\end{quote}

We expect the maximal efficiency of PSII to be the same as all of the
reaction sites are open at the start of the reaction there is no NPQ due
to the speed of the flash.

\subsection{Description of results}\label{description-of-results-2}

\begin{itemize}
\tightlist
\item
  The PSII quantum efficiency data shows no significant difference
  between the groups ({}); however, from the data, we could argue that
  there are three groups of different levels of efficiency, which would
  match up with the three types of genotype.

  \begin{itemize}
  \tightlist
  \item
    The diagnostic plots show no cause for concern, although the
    normalQ-Q plot is off slightly.
  \end{itemize}
\item
  There is a large difference in the NPQ between the three genotypes,
  with there being almost minimal NPQ in genotype B (0.38), 2.94 in C
  and almost 50\% more in genotype A ({}).

  \begin{itemize}
  \tightlist
  \item
    The residual plots show no cause for concern.
  \end{itemize}
\end{itemize}

\section{Physiological
interpretation}\label{physiological-interpretation}

The photosynthetic gas exchange results indicate a significant decrease
in carbon dioxide assimilation levels in the ssu2 mutant compared to the
wild type, supporting the hypothesis of reduced RuBisCo activity in the
mutant. Contrary to expectations, intercellular CO2 concentrations (Ci)
increased in the ssu2 mutant, suggesting complex interactions in
response to assimilation deficiencies. Stomatal conductance and
transpiration showed no significant differences between the mutant and
wild type. However, water use efficiency (WUE) decreased in the mutant,
aligning with the reduced carbon assimilation per unit of water. From
the A/Ci responses, both {} and {} were expected to decrease in the
mutant due to decreased Rubisco quantity and reduced respiration,
supporting the observed decrease in carbon fixation efficiency.

In the light use experiment, chlorophyll content was significantly lower
in the ssu2 mutant, indicating a potential feedback mechanism to
conserve resources. Electron transport rate (ETR) decreased in the
mutant, possibly due to slower proton receptor return, causing electron
transport chain backup. Maximal {} and NPQ exhibited no significant
differences, suggesting independence from RuBisCo activity.

For the light harvesting experiment, the PsbS mutant exhibited varying
levels of PSII quantum efficiency, indicating potential differences in
genotype responses. The NPQ levels differed significantly between
genotypes, with the over-expressor (genotype A) showing the highest NPQ,
aligning with the hypothesis that PsbS overexpression leads to increased
non-photochemical quenching.

In summary, the physiological interpretation suggests that the ssu2
mutant experiences compromised photosynthetic performance with reduced
carbon assimilation and altered gas exchange parameters. The light use
and harvesting experiments reveal complex interactions between genotype
and photosynthetic parameters, highlighting the intricate regulatory
mechanisms governing plant responses to genetic modifications.

\end{document}
